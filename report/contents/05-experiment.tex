\section{Experiments}

\subsection{Datasets}
This subsection should describe the datasets used in the experiments. In particular:
\begin{itemize}
  \item Dataset names and sources.
  \item Key statistics: number of samples, number of classes, image resolution, and train--validation--test splits.
  \item Any preprocessing or cleaning procedures.
  \item A brief justification of why these datasets are appropriate for evaluating the proposed method.
\end{itemize}

Our experiments are conducted on the \textbf{Visual Genome (VG)} dataset \cite{krishna2017visual}, which is the primary benchmark for Scene Graph Generation. Originally released by Stanford University, the dataset was annotated by over 33,000 Amazon Mechanical Turk workers.

\textbf{Preprocessing and Splits:} Due to the long-tail distribution and noisy annotations in the raw dataset, we utilize the widely adopted \textbf{VG-150 split} introduced by Xu et al. \cite{xu2017scene}. This preprocessing strategy filters the dataset to include only the most frequent 150 object categories and 50 relationship predicates.

\textbf{Key Statistics:}
\begin{itemize}
  \item \textbf{Data Volume:} The dataset comprises of:
        \begin{itemize}
          \item 108,077 images where each image has an average of 35 Objects, 26 Attributes, 17 Question Answers pairs and 21 Pairwise Relationships between objects.
          \item 5.4 Million Region Descriptions
          \item 1.7 Million Visual Question Answers
          \item 3.8 Million Object Instances
          \item 2.8 Million Attributes
          \item 2.3 Million Relationships
        \end{itemize}
  \item \textbf{Classes:} The vocabulary is restricted to 150 object classes and 50 relation classes.
\end{itemize}

\textbf{Justification:} Visual Genome is chosen because it provides dense, structured annotations that allow for granular evaluation of scene understanding. Unlike standard detection datasets, VG breaks images down into:
\begin{itemize}
  \item \textbf{Objects:} Localized bounding boxes for entities (e.g., \textit{man}, \textit{horse}).
  \item \textbf{Attributes:} Descriptive properties of objects (e.g., \textit{red}, \textit{tall}).
  \item \textbf{Relationships:} Directed edges describing interactions (e.g., \textit{riding}, \textit{on}).
\end{itemize}





\subsection{Evaluation Protocol}
This subsection defines how model performance is assessed. It should include:
\begin{itemize}
  \item Evaluation metrics (e.g., accuracy, AUROC, F1-score, mAP).
  \item Data splitting strategy and any cross-validation or repeated-trial setup.
  \item Baselines used for comparison and criteria for fair evaluation.
  \item Any task-specific rules or conventions followed during evaluation.
\end{itemize}

\subsection{Experimental Settings}
This subsection outlines the full implementation and training configuration for reproducibility:
\begin{itemize}
  \item Model architecture settings and important hyperparameters
        (e.g., learning rate, batch size, number of epochs).
  \item Optimization algorithms and loss functions.
  \item Hardware and software environment (e.g., GPU type, deep learning framework).
  \item Data preprocessing and augmentation pipeline.
\end{itemize}

\subsection{Results and Discussion}
This subsection presents and analyzes the experimental results:
\begin{itemize}
  \item Quantitative results summarized in tables and figures.
  \item Qualitative results when appropriate.
  \item Comparisons with state-of-the-art methods.
  \item Insights on why the proposed method performs well and any limitations or failure cases.
\end{itemize}

\subsubsection{Comparison Experiments}
This subsection reports direct comparisons with existing approaches:
\begin{itemize}
  \item Performance comparison with classical baselines and recent state-of-the-art methods.
  \item Analysis of cases where the proposed method excels.
  \item Discussion of scenarios where performance is comparable or lower.
  \item Positioning of the proposed method within the broader research landscape.
        There should be multiple tables \ref{tab1} and figures \ref{fig2} for demonstration and clarity.
\end{itemize}

\begin{table}[h]
  \centering
  \caption{Experimental results of predicate classification without Graph Constraint.}
  \label{table:pred_cls_results}
  \begin{tabular}{ |c|c c c| }
    \hline
    Category & R@20   & R@50   & R@100  \\
    \hline
    Spatial  & 0.4471 & 0.6356 & 0.7590 \\
    % \hline
    Action   & 0.2971 & 0.3835 & 0.4544 \\
    % \hline
    $mR@K$   & 0.0596 & 0.1027 & 0.1530 \\
    % \hline
    $R@K$    & 0.2991 & 0.4177 & 0.4987 \\
    \hline
  \end{tabular}
\end{table}


\begin{table}[t]
  \centering
  \caption{Table captions should be placed above the tables. Tables should be showned on top. Best values are bolded while second-best ones are underlined.}\label{tab1}
  \begin{tabular}{|p{3cm}|c|c|}
    \hline
    Methods       & Metric 1          & Metric 2          \\
    \hline
    SOTA 1 [ref]  & 0.001             & 0.002             \\
    SOTA 2 [ref]  & 0.003             & \underline{0.008} \\
    SOTA 3 [ref]  & 0.005             & 0.006             \\
    SOTA 4 [ref]  & \underline{0.007} & 0.004             \\
    \hline
    \textbf{Ours} & \textbf{0.009}    & \textbf{0.010}    \\
    \hline
  \end{tabular}
\end{table}

\begin{figure}[t]
  \includegraphics[width=\textwidth]{figures/fig1.eps}
  \caption{A figure caption is always placed below the illustration.
    Please note that short captions are centered, while long ones are
    justified by the macro package automatically.} \label{fig2}
\end{figure}

\subsubsection{Ablation Study}
This subsection evaluates the contribution of each component of the proposed method:
\begin{itemize}
  \item Removal or modification of individual modules or strategies.
  \item Measurement of performance change to quantify each component’s importance.
  \item Justification of design decisions based on empirical evidence.
\end{itemize}